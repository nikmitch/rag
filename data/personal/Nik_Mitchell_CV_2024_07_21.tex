% resume.tex
% vim:set ft=tex spell:

\documentclass[10pt,letterpaper]{article}
\usepackage[letterpaper,margin=0.75in]{geometry}
\usepackage[utf8]{inputenc}
\usepackage{mdwlist}
\usepackage[T1]{fontenc}
\usepackage{textcomp}
\usepackage{tgpagella}
\pagestyle{empty}
\setlength{\tabcolsep}{0em}

% indentsection style, used for sections that aren't already in lists
% that need indentation to the level of all text in the document
\newenvironment{indentsection}[1]%
{\begin{list}{}%
  {\setlength{\leftmargin}{#1}}%
  \item[]%
}
{\end{list}}

% opposite of above; bump a section back toward the left margin
\newenvironment{unindentsection}[1]%
{\begin{list}{}%
  {\setlength{\leftmargin}{-0.5#1}}%
  \item[]%
}
{\end{list}}

% format two pieces of text, one left aligned and one right aligned
\newcommand{\headerrow}[2]
{\begin{tabular*}{\linewidth}{l@{\extracolsep{\fill}}r}
  #1 &
  #2 \\
\end{tabular*}}

% make "C++" look pretty when used in text by touching up the plus signs
\newcommand{\CPP}
{C\nolinebreak[4]\hspace{-.05em}\raisebox{.22ex}{\footnotesize\bf ++}}

% and the actual content starts here
\begin{document}

\begin{center}
{\LARGE \textbf{Nik Mitchell}}

123B Glenmore Street\ \ \textbullet
\ \ Wellington, New Zealand 6012
\\
(+64)21 0222 9633\ \ \textbullet
\ \ nik.mitchell2@gmail.com
\end{center}

\hrule
\vspace{-0.4em}

%\subsection*{Skills}
%\begin{itemize}
%	\item Communication of technical concepts to a broad audience
%	\item Ablility to adapt to new challenges and changing circumstances
%	\item Ablility to work collaboratively as well as independently 
%	\item Very strong computer skills
%	\item Research skills
%	\item Efficient time management and prioritisation
%\end{itemize}


\hrule
\vspace{-0.4em}

\subsection*{Experience}

\begin{itemize}
  \parskip=0.1em
  
    \item
  %  \headerrow
  {\textbf{NZ Royal Commission Inquiry - COVID-19 Lessons Learned}}
  %  {\textbf{Wellington, NZ}}
  \\
  \headerrow
  {\emph{Principal Data Analyst}}
  {\emph{May 2024 -- July 2024}}
  {\emph{Created high-quality visualisations to support the Inquiry}}
  \begin{itemize*}
  	\item Created visualisations that contextualised pandemic trends (COVID-19 cases, hospitalisations, deaths and vaccinations) in New Zealand against policy decisions (e.g. lockdowns, border closing) and pandemic trends in other countries
  	\item Also conducted analyses and created visualisations to highlight the disparate impact of COVID-19 on M\a=aori and Pacific ethnic groups and people living in areas of higher socioeconomic deprivation
  	\item Worked closely with the Chair of the Commission to discuss how to tell the story of the COVID pandemic through the above visualisations in a way that draws out lessons for future pandemics
  	
  	
  \end{itemize*}
  
  
  \item
%  \headerrow
  {\textbf{COVID-19 Analytics, Ministry of Health/Health New Zealand}}
%  {\textbf{Wellington, NZ}}
  \\
  \headerrow
  {\emph{Team Lead}}
  {\emph{July 2022 -- October 2023}}
  {\emph{Managed team responsible for NZ's COVID data (cases, hospitalisations, deaths etc)}}
  \begin{itemize*}
  	\item Balanced high volumes of short-timeline requests against long-term projects in our work plan
  	\item Coordinated between analysts, management and other teams to establish clear expectations for deliverables
  	\item Incorporated Contact Tracing and Antivirals work streams into our business-as-usual responsibilities
 	\item Hired and trained new staff  	
  	\item Coordinated Intelligence stakeholders to produce daily National Situation Report for Measles outbreak
  
  \end{itemize*}
  
%  
%\item
%%\headerrow
%{\textbf{COVID-19 Analytics, Ministry of Health}}
%%{\textbf{COVID-19 Intelligence \& Surveillance, Ministry of Health}}
%%{\textbf{Wellington, NZ}}
%\\
\headerrow
{\emph{Senior Data Analyst}}
{\emph{September 2021 -- July 2022}}
{\emph{Continuing from previous role, but with additional responsibilities}}
%{\emph{Designing, conducting, visualising and automating analysis and reports related to COVID-19 in NZ (using R, SQL and Excel).}}
\begin{itemize*}
 	%	\item Interfacing with team members and other teams to coordinate timely and accurate information flow and set expectations.

	\item Automated weekly creation of visualisations for group's flagship report% as well as designing our team's standard plotting themes and pallettes for visualisations.
	\item Liased with external research groups about modelling, data requirements and data quality
	\item Collaborated with technical experts (e.g. epidemiologists) to design and refine analyses.
	\item  Created more efficient and robust data pipelines % with the targets R package,
   \item Trained and supported new team members.
\end{itemize*}


\headerrow
{\emph{Research \& Data Analyst}}
{\emph{July 2020 -- September 2021}}
{\emph{Designed, conducted, visualised and automated analysis and reports related to COVID-19 in NZ}}
%{\emph{Designing, conducting, visualising and automating analysis and reports related to COVID-19 in NZ (using R, SQL and Excel).}}
\begin{itemize*}
	\item Provided data for public, media organisations, parliament and internal stakeholders on tight timelines in emergency response conditions
	\item Communicated, resolved and prevented data issues in newly developed and constantly evolving systems
	%	\item Interfacing with team members and other teams to coordinate timely and accurate information flow and set expectations.
	\item Automated reporting processes
	\begin{itemize*}
		\item Replaced Excel-based workflows with scripted workflows in R 
		\item Automated table and graph production into Word document reports using R Markdown
		\item  Set up automatic emails for distribution of regular reporting
	\end{itemize*}
	%   \item Training and supporting new team members.
\end{itemize*}



\item
%\headerrow
{\textbf{Open Philanthropy Project, San Francisco (remote)}}
%{\textbf{Wellington, NZ}}
\\
\headerrow
{\emph{Conversation Notes Contracting}}
{\emph{February 2020 -- April 2020}}
\begin{itemize*}
	%	\item The Open Philanthropy Project is a charitable organisation that does research and grantmaking. It aims to indentify problems for which the greatest benefit can be produc ed with a given amount of money.
	\item Created and reorganised summaries of audio conversations between Open Philanthropy staff and scientific experts in AI and neuroscience
	\item The conversations were on the topic of how to estimate the computational capacity of the human brain (as an input to estimated timelines for progress in AI)

\end{itemize*}


%
%\item
%\headerrow
%{\textbf{Medical School, University of Otago, Wellington}}
%{\textbf{Wellington, NZ}}
%\headerrow
%{\emph{Receptionist}}
%{\emph{Jan 2020 -- Feb 2020}}
%\begin{itemize*}
%	\item Developed customer service mindset, engaging with students, academics and administrators.
%	\item Learned to manage and constantly reprioritise large numbers of parallel tasks to meet deadlines.
%\end{itemize*}
%
%
%\headerrow
%{\emph{IT Support Assistant}}
%{\emph{Nov 2019 -- Jan 2020}}
%\begin{itemize*}
%	\item Helping staff members with technical issues, upgrading software and hardware, securing medically sensitive data.
%\end{itemize*}
%
%
%\headerrow
%{\emph{Pathology Department Administrator}}
%{\emph{Sep 2019 -- Nov 2019}}
%\begin{itemize*}
%	   	\item General administrative tasks for the department plus acting as a personal assistant to HoD.
%	\item Gained experience working in supportive roles within teams and organising events collaboratively.
%	\item Tasks included scheduling, reimbursements, travel arrangements, spreadsheet design and maintenance
%
%\end{itemize*}

%
%\item
%%\headerrow
%{\textbf{Medical School, University of Otago, Wellington}}
%%{\textbf{Wellington, NZ}}
%\\
%\headerrow
%{\emph{Receptionist}}
%{\emph{Jan 2020 -- Feb 2020}}
%\begin{itemize*}
%%	\item Developed customer service mindset, engaging with students, academics and administrators.
%	\item Learned to manage and constantly reprioritise large numbers of parallel tasks to meet deadlines.
%\end{itemize*}
%
%
%\headerrow
%{\emph{IT Support Assistant}}
%{\emph{Nov 2019 -- Jan 2020}}
%%\begin{itemize*}
%%	\item Assisted staff members with technical issues, upgrading software and hardware, securing medically sensitive data.
%%\end{itemize*}
%
%
%\headerrow
%{\emph{Pathology Department Administrator and Assistant to Head of Department}}
%{\emph{Sep 2019 -- Nov 2019}}
%\begin{itemize*}
%%	   	\item General administrative tasks for the department and acting as a personal assistant to head of department.
%%	\item Gained experience working in supportive roles within teams and organising events collaboratively.
%	\item Tasks included inbox management, scheduling,  reimbursements, travel bookings and spreadsheet design
%
%\end{itemize*}



%  \item
%  \headerrow
%    {\textbf{University of Otago}}
%    {\textbf{Dunedin, NZ}}
%  \headerrow
%    {\emph{Demonstrator for PHSI191}}
%    {\emph{First Semester 2018}}
%  \begin{itemize*}
%    \item Helped undergraduates to develop the skills needed to gain a better understanding of the physical world.
%  \end{itemize*}
%  \headerrow
%    {\emph{Summer Studentships}}
%    {\emph{2016 \& 2017}}
%  \begin{itemize*}
%    \item Undertook $10$ week research projects for my supervisor to investigate questions of interest
%          and develop relevant skills for postgraduate study. 
%  \end{itemize*}

%  \item
%  \headerrow
%    {\textbf{Education Perfect}}
%    {\textbf{Dunedin, NZ}}
%  \\
%  \headerrow
%    {\emph{Senior Mathematics Content Generator}}
%    {\emph{Nov 2013 -- Feb 2015}}
%  \begin{itemize*}
%    \item Devised online learning resources for high school mathematics students in NZ.
%    \item Improved my ability to communicate technical concepts to a wider audience.
%    \item In charge of training and managing several other Content Generators for a couple of months 
%          towards the end of my time on the job. This included setting up collaborative spreadsheets online
%         for planning and allocating tasks as well as monitoring progress.
%   \end{itemize*}       
%

\end{itemize}


\hrule

\subsection*{Education}

\begin{itemize}
	\parskip=0.1em
	
	\item 
%	\headerrow
	{\textbf{University of Otago}}
%	{\textbf{Dunedin, NZ}}
	
	
	\headerrow
	{\emph{Masters of Physics (achieved with Distinction), Quantum Fluid Dynamics}}
	{\emph{March 2018 -- March 2019}}
%	\begin{itemize*}
%		% \item Wrote academic paper in the months following this, currently awaiting acceptance to a journal. 
%		\item Strengthened technical writing skills, programming skills and ability to work independently on long-term projects.
%	\end{itemize*}
	
	\headerrow
	{\emph{BSc. Physics (First Class Honours), Minor in Mathematics}}
	{\emph{2014 -- 2017}}
%	\begin{itemize*}
%		\item Developed strong mathematical foundations, understanding of physical systems and problem solving skills.
%	\end{itemize*}
	
	
%	\headerrow
%	{\emph{Health Sciences First Year}}
%	{\emph{2013}}
%	\begin{itemize*}
%		\item Developing a broad scientific knowledge base, including epidemiology, physics, microbiology and chemistry.
%		\item GPA of 7.6, subsequently went straight into second year physics without having taken physics in high school.
%	\end{itemize*}
		
\end{itemize}


%
%
%\hrule
%\vspace{-0.4em}
%\subsection*{Core Technical Skills}
%
%\begin{indentsection}{\parindent}
%\hyphenpenalty=1000
%\begin{description*}
%  \item[Programming:]
%  R, \ SQL, Git,\ Python,\ Excel,\  \CPP, \  \LaTeX , Vim\\
%  \item[Mathematics:]
%  Multivariable \& Vector Calculus, Linear Algebra, Real and Complex Analysis, Fourier Transforms
%\end{description*}
%\vspace{-0.4em}
%
%\end{indentsection}
%
%
%\hrule
%\subsection*{Achievements \& Interests}

%\begin{itemize*}
%  \parskip=0.1em

%  \item
%  \headerrow
%    {\textbf{Bridge}}
%    {\textbf{}}
%  \\
%  \headerrow
%    {\emph{NZ Youth Bridge Representative}}
%    {\emph{2016 -- 2018}}
%  \begin{itemize*}
%    \item I played bridge for several years and was selected to %be part of the New Zealand
%          Youth Bridge Team. 
%    \item This gave me the opportunity to travel to Australia %multiple times to take part in competitions,
%          and also attend the World Youth Bridge Championships %in Wujiang, China in August 2018.
%    \item Bridge is a game that requires sound judgement and %problem solving skills as well as interpersonal
%          skills to build and maintain a stable partnership.
%  \end{itemize*}

%  \item
%  \headerrow
%    {\textbf{Other Interests}}
%    {\textbf{}}
%  \\
%  \headerrow
%    {\emph{Things I enjoy doing and learning about in my spare time}}
%    {\emph{}}
%  \begin{itemize*}
%    \item Partner dancing (Modern Jive, West Coast Swing)
%    \item Statistics, Philosophy, Psychology, Economics
         
%  \end{itemize*}

%\end{itemize*}


\end{document}
%\grid
